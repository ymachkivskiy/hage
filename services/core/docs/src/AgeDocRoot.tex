\documentclass{mgragh} % opcje: robocza,man
\usepackage[cp1250]{inputenc}  % opcja latin2 dla Linuxa lub cp1250 dla Windows
% \usepackage{amsmath}           % �atwiejszy sk�ad matematyki
% \usepackage{amssymb} 

\usepackage[polish]{babel}
\usepackage[OT4]{fontenc}
\usepackage{polski}
%%
%%
\makeindex 

\bibliographystyle{ddabbrv}
%\nocite{*}

\begin{document}
%%
%%
%% ======== METRYCZKA PRACY ========
\title{AgE - Agentowa �rodowisko obliczeniowe}
\subtitle{wersja 2.1.0}
\author{Kamil Pi�tak, Adam Wo�}
\uczelniaNazwa{Akademia G�rniczo-Hutnicza}
\uczelniaImienia{im. Stanis�awa Staszica}
\wydzial{Wydzia� Elektrotechniki, Automatyki, Informatyki i Elektroniki }
\rok{2008}

\maketitle
%%

%%
%%
%% ======== NASZE MAKRA ========
%%

%------------------------------%
\newcommand{\id}[1]{\index{#1}}  
\newcommand{\wi}[1]{#1\index{#1}}  
\newcommand{\wwi}[1]{\emph{#1}\index{#1}}  
\newcommand{\mwi}[1]{\textbf{#1}\index{#1}}
\newcommand{\ii}[1]{\textit{#1}}
%\newcommand{}{}
%------------------------------%



%%
%% ======== SPIS TRE�CI ========
%%
\tableofcontents
%%
%% ======== STRESZCZENIE PRACY (POLSKIE) ========
\begin{licencja}
%%
%% ************ AKADEMIA G/ORNICZO-HUTNICZA W KRAKOWIE *************
%% ****************** PRACA MAGISTERSKA w LaTeX-u ******************
%%    autor: ------
%%    Copyright (C) 2002 by ------
%% ************************* Licencja *************************
%%

Rosn�ce bezrobocie w Stanach Zjednoczonych i coraz drastyczniejsze zubo�enie spo�ecze�stwa ameryka�skiego od lat jest tematem burzliwych dyskusji na �wiecie. Ochotnicze zast�py noblist�w ekonomii wielokrotnie usi�owa�y gigantycznymi dotacjami i inwestycjami o�ywi� lub raczej ,,przywo�a� z za�wiat�w'' ameryka�sk� gospodark�. Wszystkie te wysi�ki spe�z�y na niczym. Program subwencji rz�dowych dla naszego biednego alianta okaza� si� fiaskiem. Czar� przepe�ni� pi�ciokrotny wzrost nielegalnej emigracji z USA. Jednak po zast�pieniu dolara polskim z�otym, Ministerstwo Narodu postanowi�o dzia�a� bezkompromisowo i niezw�ocznie.


Pierwszym etapem pomocy rozwoju kraju ma by� budowa 243 oczyszczalni �ciek�w, do kt�rych podwodnymi wodoci�gami b�d� odprowadzane �cieki g��wnie z Europy. Pomimo, �e na przedsi�wzi�cie wydane zostan� miliony z�otych, ekonomi�ci szacuj�, �e koszty zostan� zwr�cone ju� po pierwszym roku z powodu kilka tysi�cy razy ta�szych grunt�w. Kolejnym etapem programu rozwoju ma by� ca�kowite przeniesienie wysypisk �mieci na Ziemie Sojusznika. Do tej pory powsta�o kilka takich na terenie Kalifornii, kt�re ciesz� si� du�� renom�.


- \textit{Prezesie, ludzie czekaj� a� dop�yniemy naszym statkiem-�mieciark�} - m�wi pan J�zek z firmy transportuj�cej polskie odpady - \textit{potrafi� si� rzuca� wp�aw by dosta� si� pierwsi do �adunku. Wyci�gaj� telefony, komputery, pralki, rowery, wszystko, tak �e czasem nie ma co roz�adowa�.}


W samym ministerstwie planowano stworzenie megawysypiska na Jowiszu, lecz entuzjazm by pom�c biednemu, zwyci�y�. Rozradowani Amerykanie nie kryj� euforii zachwalaj�c rozliczne profity wynikaj�ce z inwestycji:


- \textit{S�uchaj no, redaktorku, ja tu w tych �mieciach wszystko znalaz�em, i prace i sprz�t AGD, stare p�yty CD i nawet �on�\ldots To� to nies�ychane czego to ci ludzie zza oceanu nie wyrzucaj� \ldots} - chwali si� pan Rumfeldek - \textit{Rodzina m�wi, buduj stary tratw� i uciekajmy do Polski. Ale, po co mnie jecha� tam skoro mnie tu dobrze?}
\cite[GS]{GS}

%--------------------------------------------------------------------%

%%
\end{licencja}
%%

%%
%% ======== G��WNA CZʌ� PRACY ========
%%
%% ==== WST�P ====
%%
\begin{wstep}
%%
%% ************ AKADEMIA G�RNICZO-HUTNICZA W KRAKOWIE *************
%% ***************** Wydzial Matematyki Stosowanej ***************** 
%% ****************** PRACA MAGISTERSKA w LaTeX-u ******************
%%    autor: ------
%%    Copyright (C) 2002 by ------
%% ************************* Rozdzial 2 *************************
%%
%-------- Tre�� wst�pu --------%

Puste magazyny betonu, nieprzespane z nerw�w noce magazynier�w, kilkukrotny wzrost cen betonu na wszystkich rynkach handluj�cych tym strategicznym surowcem � to efekty wydarzenia, kt�re wstrz�sn�y ostatnio �wiatami betonologii i... zoologii.


- \textit{To by� szok} � t�umaczy pan Zenon, magazynier w sk�adzie betonu pod Zgierzem � \textit{Wieczorem ca�y magazyn pe�en by� betonu. Nawet troszk� na zewn�trz trzyma� musieli�my tyle tego by�o. Kiedy przyszed�em rano \ldots no mo�e w po�udnie, my�la�em, �e mam jakie� omamy wzrokowe. Zacz��em nawet niejasno wi�za� to z tymi wynalazkami na imieninach szwagra. Ale niestety � inni widzieli to samo \ldots}


Pocz�tkowo media okrzykn�y to najzuchwalsz� w historii akcj� z�odziei betonu. Syndrom pustych magazyn�w ogarn�� bowiem przesz�o po�ow� Wielkiej Rzeczpospolitej. Jakie� wi�c by�o zdziwienie wszystkich, gdy rzecznik nadzwyczajnego zespo�u kryzysowego powo�anego przez Ministerstwo S�usznych Krok�w poinformowa�, �e wszystkiemu winne s� \ldots


\textbf{\ldots korniki}.


Jak podczas konferencji prasowej wyja�nia� rzecznik:


- \textit{Nasze dochodzenie wykaza�o, �e znikaj�cy beton to dzie�o nowego gatunku kornika. Po prostu ludzie przestali budowa� z drewna, a podstawowym od jakiego� czasu budulcem sta� si� beton. No i zadzia�a�a ewolucja. Natura nie lubi pr�ni \ldots}


- \textit{Ju� od jakiego� czasu obserwowali�my, �e te ma�e gnojki zawzi�cie eksperymentuj�.} � dodaje Cezary Korniszon, dyrektor O�rodka Bada� Nad Kornikami (tak tak, dobrobyt w Wielkiej Rzeczpospolitej jest tak wielki, �e nawet na takie przedsi�wzi�cia nie brakuje pieni�dzy) w K�rniku � \textit{Ju� rok temu zanotowali�my przypadek kolonii kornik�w usilnie pr�buj�cych od�ywia� si� r�nymi metalami. Niestety kiepsko im sz�o\ldots permanentna sra\ldots ekhm\ldots niestrawno�� i takie tam \ldots og�lnie nie by� to mi�y widok. Widocznie teraz dosz�y do wniosku, �e beton jest lepszy.}


To �e beton jest lepszy od wszystkiego innego w WRP wszyscy wiedz� od dawna, ale gigantyczny deficyt betonu sprawi�, �e nie wszyscy byli szcz�liwi, �e korniki dosz�y do podobnych przemy�le�. Reakcja Ministerstwa S�usznych Krok�w na wnioski komisji by�a natychmiastowa. Departament Ochrony �rodowiska, prowadz�cy rejestr zagro�onych gatunk�w, dosta� polecenie dopisania do listy \textbf{Kornika Betoniarza} i do�o�enia wszelkich wysi�k�w by jego znalezienie si� na tej li�cie by�o jak najbardziej uzasadnione. Prace nad tym maj� si� rozpocz�� jak tylko opracowana zostanie trutka odpowiednio komponuj�ca si� smakowo z zar�wno p�ynnym jak i st�a�ym betonem.
Chodz� wprawdzie pog�oski o protestach obro�c�w �rodowiska domagaj�cych si� ochrony dla tego gatunku, ale jak powiedzia� anonimowy przedstawiciel w�adz:


- \textit{Kto by si� przejmowa� lud�mi, kt�rzy wol� �y� po lasach w drewnianych sza�asach, zamiast mieszka� w zdrowym, przytulnym i przyjaznym cz�owiekowi betonowym �rodowisku.} \cite[KB]{KB}
%%
\end{wstep}
%%
%% ==== ROZDZIA� 1 ====
%%
%% A tutaj tak dla przyk�adu jest \part
\part{Musztarda kotwiczna}

\chapter{Wprowadzenie do przypraw walcowych}
%%
 \input{AgeDoc_Ch1}
%%
\mgrclosechapter
%%
%% ==== ROZDZIA� 2 ====
%%
 %% ************ AKADEMIA G/ORNICZO-HUTNICZA W KRAKOWIE *************
%% ***************** Wydzial Matematyki Stosowanej ***************** 
%% ****************** PRACA MAGISTERSKA w LaTeX-u ******************
%%    autor: ------
%%    Copyright (C) 2002 by ------
%% ************************* Rozdzial 2 *************************
%%

\chapter{Informacje dotycz�ce eh\ldots}

\section{Informacje og�lne}

$$ e^{\pi i} + 1 = 0$$

\mgrclosechapter
%%
%%
%% ======== BIBLIOGRAFIA ========
%%
%%% ************ AKADEMIA G/ORNICZO-HUTNICZA W KRAKOWIE *************
%% ***************** Wydzial Matematyki Stosowanej *****************
%% ****************** PRACA MAGISTERSKA w LaTeX-u ******************
%%    autor: ------
%%    Copyright (C) 2002 by ------
%% ******************** Plik z literatura *************************
%%

\bibliography{MgrRoot}

%%\begin{thebibliography}{88}
%%\normalsize
 %%\addtolength{\itemsep}{\smallskipamount} % tak dla przejrzystosci
%% \bibitem{LIT1} Xxx A.: \textit{Tytu� 1} Wyd. 1. Miasto 1: Wydawnictwo 1, 1888. 
%% \bibitem{LIT2} Aaa B.: \textit{Tytu� 2}. Miasto 2: Wydawnictwo 1, 1777.  
%% \bibitem{LIT3} Bbb C.: \textit{Tytu� 3}, t. 1. Miasto 3: Wydawnictwo 1, 1666.
%% \bibitem{LIT4} Ccc D.: \textit{Tytu� 4}. Miasto 4: Wydawnictwo 2, 1555.
%% \bibitem{LIT5} Ddd E.: \textit{Tytu� 5}. Miasto 5: Wydawnictwo 3, 1333.
%%\end{thebibliography}



%%
%% ======== DODATKOWE ELEMENTY (nieobowi�zkowe) ======== 
%%
%\printindex  
%%

\end{document}

